\documentclass{revtex4}

\usepackage{amsmath,amssymb}
\usepackage{graphicx}
\graphicspath{{figures/}}

\usepackage{color}
\newcommand{\jb}[1]{{\color{blue}\bf #1}}

\newcommand{\proj}{\ensuremath{\mathcal{P}}}
\newcommand{\kvec}{{\bf k}}
\newcommand{\xvec}{{\bf x}}

%%% Stress energy tensor
\newcommand{\opTT}[1]{\ensuremath{#1}^{{\rm TT}}}
\newcommand{\tij}{\ensuremath{\opTT{T_{ij}}}}
\newcommand{\tijNoT}{\ensuremath{\mathcal{T}_{ij}}}
\newcommand{\prof}{\ensuremath{A}}

%%% Discrete wavenumbers
\newcommand{\kzero}{\ensuremath{\tilde{k}}}
\newcommand{\kone}{\ensuremath{\kappa}}
\newcommand{\ktwo}{\ensuremath{\bar{k}}}

%%% Parameters for Gaussian blob source
\newcommand{\aSym}{\ensuremath{\mathcal{A}}}
\newcommand{\aAsym}{\ensuremath{\mathcal{B}}}


\title{Artificial Production of Gravitational Waves by Localised Coherent Sources}

\begin{document}
\maketitle

\section{Introduction}
In this paper we will study the effects of inconsistent notions of differentiation between the computing the source term from a given underlying field distribution
This is the relevant case for simulations, since it is the field $\phi_i$ that are evolved directly, not the stress-energy tensor that acts as the source for the gravitational wave perturbations.
In this paper, we will not worry about how the dynamics of the fields themselves are modified by the choice of discrete stencil, but will isolate the effects of the nonlinear transformation from $\phi$ to the stress-energy tensor, followed by the (linear) projection onto TT modes of the gravitational waves.

\section{Pointless Notes}
Rough outline of some steps to take:
\begin{itemize}
\item
  \begin{itemize}
  \item Assume spherical Gaussian, compute the numerically computed GW signal using various combinations of discrete stencils for the source and three-metric
  \item Same calculation with an aspherical Gaussian
  \item Same calculation with a Gaussian with spherical harmonic perturbations to radius
  \item General formalism that will reduce to various convolutions of the FT of the object
  \end{itemize}
\item Once this is done, translate the numerical error in the source into a numerical error in the calculated GW spectrum using some appropriate Green's functions.  This will basically assume that the time-integrator is perfect, so that all errors are from spatial discretisation
\item Think a bit about what time integration errors do (for long time runs lots of time-stepping algorithms will will constantly violate things like energy conservation, which will modify the amplitude of $h_{ij}$ and in long time runs this can accumulate.
\end{itemize}

Two sources of error
\begin{itemize}
\item Incorrect projection of stress-energy tensor
\item Incorrect projection of stress-energy tensor onto $h_{ij}$
\end{itemize}
The latter will be highly entangled with the nonlinear dynamics of the scalar fields, although it is very difficult to see how the discretisation errors will cancel out unless you are very careful.
The second effect is really just an issue of how you define linear operators.

\section{Numerical Approach to Extraction of Gravitational Waves}
In the standard approach, calculation of gravitational waves sourced by stongly nonequilibrium and inhomogeneous scalar field dynamics is done in a two step process (for an exception see~\cite{Huang:HLattice,Bastero-Gil}.
Here we briefly outline this approach, restricting ourselves to the case of canonical kinetic terms for the scalar fields for simplicity.

First, the scalar field dynamics is evolved from set specified initial condition using
\begin{equation}
  \label{eqn:scalar-eom}
  \ddot{\phi}_i + 3H\dot{\phi} - \nabla^2\phi + \frac{\partial V}{\partial \phi_i} = 0 \, .
\end{equation}
Depending on the application, the Hubble parameter may be externally imposed or else solved self-consistently.
\jb{Comment that laplacian and grad squared are already constrained.}
The standard approach then considers the linear response of the gravitational perturbations to these dynamically evolving fields, ignorning all backreaction and rescattering effects of the metric perturbations.
Technically, the metric is written in synchronous gauge as
\begin{equation}
  ds^2 = -dt^2 + a^2(t)(\delta_{ij} + h_{ij})dx^idx^j
\end{equation}
and the solution of~\eqref{eqn:scalar-eom} is treated as an external input to evolve the transverse traceless projection of $h_{ij}$
\begin{equation}
  \label{eqn:sourced-h-tt}
  \ddot{h}_{ij}^{TT} + 3H\dot{h}_{ij}^{TT} - \nabla^2h_{ij}^{TT} = \frac{1}{a^2}\left[\partial_i\phi\partial_j\phi\right]^{TT}
\end{equation}
where ${}^{TT}$ indicates the transverse-traceless (TT) projection of a given tensor.
\jb{Write metric perturbations}
In the continuum, this is defined by
\begin{equation}
  \label{eqn:tt-proj}
  S_{ij}^{TT} = \proj_{ij,lm}^{TT}S_{lm}
\end{equation}
with
\begin{equation}
  \label{eqn:proj-TT}
  \proj_{ij,lm} = P_{il}P_{jm} - \frac{1}{2}P_{ij}P_{lm}
\end{equation}
and
\begin{equation}
  \label{eqn:proj-simple}
  P_{ij} = \delta_{ij} - \hat{k}_i\hat{k}_j
\end{equation}
for an arbitrary symmetric tensor $S_{ij}$.
In practice,~\eqref{eqn:sourced-h} is rarely solved directly.
Instead, the unprojected equation
\begin{equation}
  \label{eqn:sourced-h-full}
  \ddot{h}_{ij} + 3H\dot{h}_{ij} - \nabla^2h_{ij} = a^2\partial_i\phi\partial_j\phi
\end{equation}
with $h_{ij}^{TT}$ obtained by applying the TT projection to the resulting solution.
Furthermore, the projection is usually done in Fourier space, with $h_{ij}$ evolved in real space using some discrete approximation to the Laplacian operator.
\emph{In the continuum}, these two approaches are equivalent. \jb{Check discrete case for commutativity.}

Of course, to put these equations on a finite computer, discretization choices must be made for both the time-coordinate and the spatial coordinates.
Our focus in this paper will be on the effects of spatial discretization on the extracted signal, so for the purposes of illustration we will assume a time-integrator with perfect accuracy.
Examples which have been demonstrated to achieve $\mathcal{O}(10^{-14})$ accuracy in nonlinear scalar field dynamics after inflation and during first-order phase transitions include symplectic Yoshida~\cite{} and Gauss-Legendre~\cite{} schemes.
The former has the advantage of being explicit and not requiring temporary storage of copies of the entire lattice.
At any rate, for any reasonable choice of integration scheme, improved temporal accuracy can be achieved through decreased time step at the cost of additional CPU cycles.
Improved spatial resolution, however, requires either implementation of an improved discretization scheme, or else an increase in the amount of memory to store the simulation data.
This is a far more severe computational burden, so in many applications the spatial convergence is the key consideration.
Furthermore, maintaining an equivalent level of temporal accuracy requires holding the Courant step $dx/dt$ fixed, so a decreased grid-spacing must be accompanied by a corresponding decrease in $dt$. {\bf some comment about conditions for this to be true.  When lattice sims are needed and all modes are excited and interacting this is usually true.}

Examining the preceding formalism, we see that discrete derivatives must be defined in four distinct places:
\begin{enumerate}
\item the Laplacian acting on $\phi_i$,
\item the Laplacian acting on $h_{ij}$,
\item the derivatives in $T_{ij}$, and
\item in the projection operator.
\end{enumerate}
Although not obvious from the equations of motion, these definitions are all intrinsically linked to each other, and systematic errors will be introduced if the various derivative operators are not properly related to each other.
In this paper, we will explicitly quantify these errors and examine the consequences for the case of long-lived scalar field configurations with fixed frequncies.
A particular example of relevance to the recent literature is the case of oscillons produced after inflation.

\section{TT Projections and the SVT Decomposition of a 3-tensor}
Given a symmetric matrix $S_{ij}$, the transverse traceless (TT) projection can be obtained by applying the projection operator operator
\begin{align}
  \label{eqn:tt-proj-fourier}
  \proj^{TT}_{ij,lm} &= P_{il}P_{jm} - \frac{1}{2}P_{ij}P_{lm} \\
  P_{ij} &= \delta_{ij} - \hat{k}_i\hat{k}_j \, .
\end{align}
Here, we have expressed the projector in Fourier space, which assumes that the action of a derivative operator in Fourier space can be obtained by the replacement $\partial_i \to ik_i$.
As we will see shortly, this naive replacement fails for a wide range of numerical approximation schemes based on lattice discretisations of a continuous system.
Below we will derive the appropriate real space version of the TT projector~\eqref{eqn:tt-proj-fourier}, and it's generalisation to Fourier space.

To derive~\eqref{eqn:tt-proj-fourier}, it is convenient to consider the standard SVT decomposition of an arbitrary symmetric matrix $S_{ij}$
\begin{equation}
  S_{ij} = \frac{S}{3}\delta_{ij} + \left(\partial_i\partial_j - \frac{1}{d}\delta_{ij}\nabla^2\right)S^{(S)} + (\partial_iS_j^{(V)} + \partial_jS^{(V)}_i) + S_{ij}^{(TT)} \, .
\end{equation}
For notational convenience, we also define the trace-free projection
\begin{equation}
  \bar{S}_{ij} = S_{ij} - \frac{\delta_{ij}}{d}S \qquad S = S_{ii} \, .
\end{equation}
The various components can be obtained from the original matrix through the projection operators
\begin{subequations}
\begin{align}
  \proj^{Tr,\parallel}_{ij,lm} &\equiv \frac{1}{d}\delta_{ij}\delta_{lm} \\
  \proj^{(S),\parallel}_{ij,lm} &\equiv \frac{d}{d-1}\left(\partial_i\partial_j - \frac{1}{d}\delta_{ij}\nabla^2\right)\nabla^{-4}\left(\partial_l\partial_m - \frac{1}{d}\delta_{lm}\nabla^2\right) \\
  \proj^{(S),\perp}_{ij,lm} &= \delta_{il}\delta_{jm} - P^{(S),\parallel}_{ij,lm} \\
  \proj^{(V),\parallel}_{ij,lm} &\equiv (\delta_{ia}\partial_j + \delta_{ja}\partial_i)\nabla^{-2}\partial_b\left(\delta_{al}\delta_{bm} - \proj^{(S),\parallel}_{ab,lm} - \proj^{Tr,\parallel}_{ab,lm}\right) \\
  \proj^{(V),\perp}_{ij,lm} &= \delta_{il}\delta_{lm} - P_{ij,lm}^{(V),\parallel} \\
  \proj^{(T),\parallel}_{ij,lm} &= \\
\end{align}
\end{subequations}
Here ${}^\parallel$ indicates the projection onto solely the given component of the matrix.
We obtain the TT projection through repeated removal of individual components
\begin{align}
  \mathcal{P}_{ij,lm}^{(TT)} &= \left(\delta_{ia}\delta_{jb} - \proj^{(V),\parallel}_{ij,ab}\right)\left(\delta_{am}\delta_{bn}-\proj_{ab,mn}^{(S),\parallel}\right)\left(\delta_{ml}\delta_{nm}-\proj_{mn,lm}^{Tr,\parallel}\right) \\
  &=  \delta_{il}\delta_{jm} - \proj_{ij,lm}^{(V),\parallel} - \proj_{ij,lm}^{(S),\parallel} - \proj_{ij,lm}^{Tr,\parallel}
\end{align}

The scalar potential ($S$) and vector potential $S_i$ are easily obtained by solving
\begin{subequations}
  \begin{align}
    \nabla^4 S^{(S)} &= \frac{d}{d-1}\partial_i\partial_j\bar{S}_{ij} \\
    \nabla^2 S_i &=
  \end{align}
\end{subequations}

We now assume we work in the continuum, and that we can treat the system as period so the inverse Laplacian operators are easily expressed in terms of wavenumbers.
To extract the total scalar contribution (the trace and $S^{(S)}$), we can use the projection
\begin{equation}
   \proj^{{\rm scalar}}_{ij,lm} = \proj_{ij,lm}^{(S),\parallel} + \proj_{ij,lm}^{Tr,\parallel} = \frac{1}{2}P_{ij}P_{lm} + \hat{k}_i\hat{k}_j\hat{k}_l\hat{k}_m \, .
\end{equation}
Except for the term involving 4 momenta, this is the second term in the representation~\eqref{eqn:}.
To obtain the projector onto the vector modes, first define
\begin{equation}
  Q^{(V)}_{ij,lm} = \hat{k}_i\hat{k}_l\delta_{jm} + \hat{k}_j\hat{k}_l\delta_{im}
\end{equation}
which satisfy
\begin{equation}
  Q^{(V)}_{ij,ab}\proj^{{\rm scalar}}_{ab,lm} = 2\hat{k}_i\hat{k}_j\hat{k}_l\hat{k}_m
\end{equation}
from which we immediately obtain (using the symmetry of the projector)
\begin{equation}
  \proj_{ij,lm}^{(V)} = \hat{k}_i\hat{k}_l\delta_{jm} + \hat{k}_j\hat{k}_m\delta_{il} - 2\hat{k}_i\hat{k}_j\hat{k}_l\hat{k}_m = -P_{il}P_{jm} + \delta_{il}\delta_{jm} - \hat{k}_i\hat{k}_j\hat{k}_l\hat{k}_m \, .
\end{equation}
For convenience in future calculations, we write explicitly
\begin{subequations}
\begin{align}
  S_{ij}^{\rm scalar} &= \frac{\delta_{ij}}{2}\left(S_{ll} - \hat{k}_lS_{lm}\hat{k}_m\right) - \frac{\hat{k}_i\hat{k}_j}{2}\left(S_{ll} - 3\hat{k}_lS_{lm}\hat{k}_m\right) \\
  S_{ij}^{(V)} &= \hat{k}_i\hat{k}_lS_{lj} + \hat{k}_j\hat{k}_lS_{li} - 2\hat{k}_i\hat{k}_j\hat{k}_lS_{lm}\hat{k}_m \\
  S_{ij}^{(T)} &= S_{ij} - \hat{k}_iS_{jl}\hat{k}_l - \hat{k}_jS_{il}\hat{k}_l + \frac{1}{2}\delta_{ij}\left(\hat{k}_lS_{lm}\hat{k}_m - S_{ll}\right) + \frac{1}{2}\hat{k}_i\hat{k}_j\left(S_{ll}+\hat{k}_lS_{lm}\hat{k}_m\right) \\
\end{align}
\end{subequations}
from which we see we require the following three contractions $S_{ii}$, $\hat{k}_lS_{li}$ and $\hat{k}_lS_{lm}\hat{k}_m$ of $S_{ij}$.
It is also straightforward to verify that $S_{lm}^{(T)} = \bar{S}_{lm}^{(T)}$ so that we are free to first remove the trace of $S_{ij}$ if we want.

Using these expressions, we immediately obtain the original result for the TT projector,
but now we have clarified the connection between the full SVT decomposition of a symmetric matrix $S_{ij}$ and the projection onto the TT components.
Furthermore, it is straightforward to see
\begin{equation}
  \proj_{ij,lm}^{(T)}\proj_{ij,ab}^{(T)} = \proj_{lm,ab}
\end{equation}
which allows us to express
\begin{equation}
  S_{ij}^{(T)}S_{ij}^{(T)} = S_{ij}S_{ij} - 2\hat{k}_iS_{il}S_{lj}\hat{k}_j + \left(\hat{k}_iS_{ij}\hat{k}_j\right)^2 - \frac{1}{2}\left(S_{ii}-\hat{k}_iS_{ij}\hat{k}_j\right)^2
\end{equation}
in terms of the four scalar quantities $\mathrm{Tr}(S), \mathrm{Tr}(S^2), \hat{k}^TS\hat{k}$, and $\hat{k}^TS^2\hat{k}$.
As we will see below, this quantity is of central importance when computing the GW spectrum from a long-lived source with fixed frequency.

{\bf Add expressions for these in spherical instead of Cartesian coordinates.  Also add the corresponding transformed tensor components.  Check if these expressions are covariant or not.}

However, despite this apparent simplicity, it should be clear that caution must be exercised when working with a discretised system, in particular when the derivative operators appearing in the projection are discretised by means other than pseudospectral or spectral methods.
{\bf Add a comment that in real space this ambiguity arises because we're inverting differential operators, which on a lattice is just a huge matrix inversion}
To see how problems may arise, notice that to extract the trace-free scalar part of $S_{ij}$, we must invert the operator $\nabla^4$.
For periodic systems with exact derivatives, this is easily done in Fourier space by simply dividing by $k^4$.
However, when finite-difference approximations are used, the operator inversion becomes a large matrix inversion problem.
The key point is that the eigenvalues of the discrete operator approximation $\nabla^4$ will, in general, differ from the continuum values.
For example, with the stencil
{\bf write the general stencil}
we have
{\bf write what this looks like.  Product of sines and cosines.}

\section{Gravitational Waves from Sources of Fixed Frequency}
\jb{Go through and check all the FT conventions and normalisations that are used here.  Need these in order to get the correct scaling of the source.}
\jb{Generalize this to $\omega(t)$ instead of linear in time}
We first consider GWs produced in an expanding FRW background.
\jb{Fill in a brief summary of the derivation.  Useful for extension to discrete Laplacian operator.}
Using Green's function techniques, we can show the fractional energy density in gravitational waves per comoving wavenumber is
\begin{equation}
  \Omega_k^{GW}({\bf k}) = \frac{1}{3M_P^2H^2}\frac{d\ln\rho_{\rm GW}}{d\ln k} =  \frac{1}{3M_P^2H^2}\sum_{ij}\left|\dot{h}^{\rm TT}_{ij}\dot{h}_{ij}^{\rm TT}\right|^2 \approx \frac{S_k}{3M_P^2H^2}
\end{equation}
with
\begin{equation}
  S_k = \frac{k^3}{2M_P^2Va^4}\int d\Omega_k\sum_{ij}\left(\left|\int_{\tau_i}^\tau d\tau'\cos(k\tau')\cos^2(\omega \tau')\tij({\bf k},\tau')\right|^2 + \left|\int_{\tau_i}^\tau d\tau'\sin(k\tau')\tij({\bf k},\tau')\right|^2 \right)
\end{equation}
where $\tau$ is the conformal time and $k$ the comoving momentum.
Here the energy density in gravitational waves is
\begin{equation}
  \rho_{\rm GW} = \frac{M_P^2}{4}\left\langle\dot{h}_{ij}^{\rm TT}\dot{h}_{ij}^{TT}\right\rangle_V
\end{equation}
where $\langle\cdot\rangle_V$ indicates a spatial average over several wavelengths of the tensor perturbations.

We must now adopt a model for the source $\tij$.
To explore the effects of a single localized source with fixed frequency (in cosmic time), we assume
\begin{equation}
  \phi({\bf x},t) = \prof({\bf x},t)\cos(\omega t)
\end{equation}
with $\prof$ some profile function with non-oscillatory and smooth temporal behavior.
We thus have
\begin{equation}
  \tij({\bf x},t) = \tijNoT({\bf x},t)\cos^2(\omega t) \, .
\end{equation}
\jb{This isn't specific to a localized source yet.}
The extension to multiple harmonics and multiple well-separated objects will be straightforward and considered below.

We have
\begin{subequations}
\begin{align}
  \cos k\tau\cos^2\omega t &= \frac{1}{4}\cos(2\omega t-k\tau) + \frac{1}{4}\cos(2\omega t+k\tau) + \frac{1}{2}\cos k\tau \\
  \sin k\tau\cos^2\omega t &= \frac{1}{4}\sin(2\omega t-k\tau) + \frac{1}{4}\sin(2\omega t+k\tau) + \frac{1}{2}\sin k\tau \, .
\end{align}
\end{subequations}
The dominant contributions to the integrals come from the sinusoids of $2\omega t - k\tau$, which have stationary points in their evolution when
\begin{equation}
  k = 2\omega\frac{dt}{d\tau} = 2\omega a
\end{equation}
Assuming smooth (non-oscillatory) behaviour in $a$ and $\mathcal{T}_{ij}^{TT}$, we can approximate
\begin{equation}
  \int_{\tau_i}^\tau d\tau' a(\tau')\mathcal{T}_{ij}({\bf k},\tau')\cos k\tau'\cos^2\omega t \approx \sqrt{2\pi i}\frac{a_k\mathcal{T}_{ij}({\bf k},\tau_k)}{4\sqrt{2\omega a^2_kH_k}}\cos(2\omega t_k -k\tau_k)\Theta(\tau - \tau_k)
\end{equation}
and
\begin{equation}
  \int_{\tau_i}^\tau d\tau' a(\tau')\mathcal{T}_{ij}({\bf k},\tau')\sin k\tau'\cos^2\omega t \approx \sqrt{2\pi i}\frac{a_k\mathcal{T}_{ij}({\bf k},\tau_k)}{4\sqrt{2\omega a^2_kH_k}}\sin(2\omega t_k -k\tau_k)\Theta(\tau - \tau_k)
\end{equation}
where $\Theta(x)$ is the Heaviside step function.
Notice the tremendous simplification that occurs in the temporal integrals due to the narrow and fixed bandwidth of the source.
If the source instead contains many frequencies, then interference between GWs produced at different times can be significant and the source will not in general factor out of the expression.

Here the subscript ${}_k$ indicates that the given quantity is evaluated at the conformal time $\tau_k$ when
\begin{equation}
  2\omega\frac{dt}{d\tau} = k \qquad a(\tau_k) = \frac{k}{2\omega} \, .
\end{equation}
Squaring and summing, we find
\begin{equation}
  2M_P^2Va^4S_k \approx k^3\frac{\pi}{16 \omega H_k}\sum_{ij}\left|\mathcal{T}_{ij}({\bf k},\tau_k)\right|^2\Theta(2\omega a - k) \, .
\end{equation}
Now, suppose $a = a_0 t^p$, then we have $H_k^{-1} = \frac{k^{1/p}}{p(2\omega)^{1/p}}$
{\bf be more careful with the $a_0$ here.}

We immediately note there is a sharp cutoff in the spectrum at $k = 2\omega a$.
Moving beyond the approximation used here, the stationary phase contribution will have a short period of time during which it turns on (of width $2\omega a'$), which smooths the step function slightly, but still leave a rapidly decreasing function beyond the cutoff.
The structure for smaller $k$ values depends on the scaling properties of the source.
If the sources scales as a power law in the scale factor $\mathcal{T}_{ij} \sim a^\mu$, then we end of with $S_k \sim k^{3+\frac{1}{p}+\mu}$.
If particular, if the fields have a fixed physical size, then in real space $T_{ij} \propto a^2$ and the Fourier transform is scales as $a^{-1}$.
Assuming an equation of state parameter $w$, we then obtain as spectral index
\begin{equation}
  n_G = \frac{5}{2} + \frac{3}{2}w \, .
\end{equation}
Furthermore, the width of region over which the source turns on scales as $2\omega a^2H \sim a^{\frac{1}{2}\left(1-3w\right)}$.

\subsection{Scaling Profile Solutions}
Given a scaling form for the profile
\begin{equation}
  \prof(\xvec,t) = a^\alpha \prof_0(a^\beta\xvec)
\end{equation}
we obtain the following scaling relationship for $\tijNoT$
\begin{equation}
  \tijNoT(\xvec,t) = a^{2\alpha+2\beta}\tijNoT(a^\beta\xvec)
\end{equation}
and for its Fourier transform
\begin{equation}
  \tilde{\tijNoT}(\kvec,t) = a^{2\alpha-\beta}\tilde{\tijNoT}_{,0}(\kvec) \, .
\end{equation}
where we have assumed three spatial dimensions in the scaling of the Fourier transform.
As an important special case, $\beta=1$ corresponds to a field profile of fixed physical size.

Denoting the time when the scaling of the profile begins by $\tau_i$, we obtain the following
\begin{equation}
  \Omega_{\rm GW} = A_{\rm GW}\left(\frac{k}{2\omega}\right)^{\rm n_{\rm GW}}\Theta(k-k_i)\Theta(2\omega a - k))
\end{equation}
with
\begin{equation}
  n_{\rm GW} = \frac{5}{2} + \frac{3}{2}w + - 2(\beta-1) + 4\alpha
\end{equation}
where we have used the fact that $H \sim a^{-\frac{3}{2}(1+w)}$ (and $p = \frac{2}{3(1+w)}$) for expansion driven by a fluid with constant equation of state $w$.
With the case of a collection of oscillons in mind, we notice that for field profile of fixed physical size ($\beta=1$) and matter dominated expansion ($w=0$), a spectral slope greater that $\frac{5}{2}$ requires the presence of a \emph{growing} source.
\jb{Fix the step functions.}

\jb{Generalize this to time-dependent $\omega$ since the derivation is essentially unchanged}
\jb{Generalize the scaling to two arbitrary functions instead of $\alpha$ and $\beta$ for the since frequency case.}

\jb{As it relates to the PRL, discretum effects cause the cutoff in the spectrum to move to the right, and since the source is presumably increasing in amplitude as the finite-differencing effects increase, the slope of the ``peak'' will also increase.  This is clearly what qualitatively happens in the PRL plot.  This is of course ignoring errors in time-stepping.}

\subsection{Generalisation to a collection of separated local sources}
The above result generalises almost trivially to the case of a collection of $N$ localized sourced in the field $\phi$
\begin{equation}
  \phi = \sum_{n=1}^{N} P_n({\bf x-x}_n,t)\cos\left(\omega_n(t-t_n)\right)
\end{equation}
where the $P_n$'s are some set of profile functions and ${\bf x}_n$ label the centers of the local blob of field configuration.
Assuming that $P_{n_1}({\bf x-x}_{n_1},t)P_{n_{2}}({\bf x - x}_{n_2},t) \ll 1$ for $n_1 \neq n_2$ so that the interactions between the various blobs can be ignored, we find
\begin{equation}
  T_{ij}({\bf x}) = \sum_{n=1}^N\partial_iP_n({\bf x-x}_n)\partial_jP_n({\bf x-x}_n)\cos^2(\omega_n(t-t_n))
\end{equation}
with Fourier transform
\begin{equation}
  \tilde{T}_{ij}({\bf k}) = \sum_{n=1}^Ne^{i{\bf k}\cdot {\bf x}_n}\cos^2(\omega_n(t-t_n))
\end{equation}
{\bf Add the appropriate delta functions, then mention we have to average over a few wavelengths for this to make sense}

\section{Gravitational Waves in the Discretum}
Having established the general formalism for production of gravitational waves by long-lived scalar sources, we now investigate how various finite-differencing approximations may influence numerically obtained results.
In this work, we will isolate the effects independently of each other as much as possible.
Of course, in an actual application this will typically not be the case, and various discretization errors will interact with each other in (typically nonlinear) ways.
A precise characterization of the theoretical error induced by typical discretization procedure thus requires a holistic approach.
However, the unified approach also mixes various distinct effects together and makes it difficult to discern how various approximations feed into the computed GW spectrum.
Therefore, to provide the reader with as much intuition as possible, we follow a modular approach here.
In principle, combining the analyses presented below will allow for a complete analysis of a given numerical setup.

We will consider the effects of the numerical approximation to the Laplacian, and the interplay between the differentiation operators used to define $T_{ij}$ and the TT projection operators.
For the case of long-lived field configurations, in particular, we will find that a cavalier attitude towards the relationship between derivative operators used to define field derivatives $\partial_i\phi$ and projection operators can lead to artificial gravitational wave power leaked from the physical scalar sector of the true stress-energy tensor.
Such artificial leakage can lead to radically incorrect inferences about the produced $\Omega_{\rm GW}$ since the scalar sector of $T_{ij}$ is many orders of magnitude larger than the tensor sector, so even a small fractional numerical leakage can easily swamp the true signal.
The results obtained above make no direct assumptions about the physical vs. numerical nature of the source $\tij$, so we will be able to apply them in the following with essentially no modification.

\subsection{Discrete Laplacian Evolution}
Before considering the projection of $\partial_i\phi\partial_j\phi$, we first consider the interaction between the TT projection and the choice of Laplacian operator to evolve $h_{ij}$ with.
In order to directly connect with common numerical approach, and contrast them with the continuum formulation, here we will briefly derive the necessary modifications to the formalism above for the case when $h_{ij}$ is evolved using a finite-differencing approximation for the Laplacian.

We continue to work in conformal time
\begin{equation}
  h_{ij}'' + 2\mathcal{H}h_{ij}' - L[h_{ij}] = \frac{2}{M_P^2}a^2\pi_{ij}
\end{equation}
where we have now replaced $\nabla^2$ with its discrete lattice representation
\begin{equation}
  L[f](x_i) \equiv \frac{1}{dx^2}\sum_\alpha c_{\alpha}\left(f(x_{i+\alpha})-f(x_i)\right) \, .
\end{equation}
As before, it is convenient to instead work with $\bar{h}_{ij} \equiv ah_{ij}$
\begin{equation}
  \bar{h}_{ij}'' - L[\bar{h}_{ij}] - \frac{a''}{a}\bar{h}_{ij} = \frac{2}{M_P^2}a\pi_{ij} \, .
\end{equation}
Fourier transforming, we obtain
\begin{equation}
  \bar{h}_{ij}''(\kvec) + \left(k_{\rm eff}^2(\kvec) -\frac{a''}{a}\right) \bar{h}_{ij}(\kvec) = \tij(\kvec) 
\end{equation}
with the effective wavenumber
\begin{equation}
  k_{\rm eff}^2({\bf k}) = \sum_\alpha c_\alpha \, .
\end{equation}
\jb{Fill in equation with cosines and convert to angular variables.}
\jb{Write out Green's function solution, project it, and comment on how the conversion of a fixed $T_{ij}$ is converted into $h_{ij}$.  To isolate effect, can just pretend that the $T_{ij}^{TT}$ projection is exact.}

Finally, for the gravitational wave power, we find that~\eqref{eqn:} is replaced by
\begin{equation}
  S_{\bf k} = \frac{k^3}{2M_P^2Va^4}\int d\Omega_k\sum_{ij}\left(\left|\int_{\tau_i}^\tau d\tau'\cos(k_{\rm eff}\tau')\cos^2(\omega \tau')\tilde{\mathcal{T}}^{TT}_{ij}({\bf k},\tau')\right|^2 + \left|\int_{\tau_i}^\tau d\tau'\sin(k_{\rm eff}\tau')\cos^2(\omega \tau')\tilde{\mathcal{T}}^{TT}_{ij}({\bf k},\tau')\right|^2 \right)
\end{equation}
\jb{Check the above to make sure I didn't implicitly drop any angular dependence horrible angular dependence.}

\subsection{Projection of $T_{ij}$}

\subsection{Grid Based Evaluation: Binning}
\jb{Go through how the implicit binning induced by the discrete k grid folds into the final answer.}

\subsection{Numerical versus actual signal}
{\bf Rewrite this to avoid the intermediary of $h_{ij}$ and simply insert the source into the integrals.  Basic idea stays the same.}
Thus far, we have made no assumption about whether the source $T_{ij}$ arises from a physical mechanism or from numerical erros.  Let's now briefly consider how the discretisation errors outlines above feed into the numerically inferred GW amplitude.
We write
\begin{equation}
  \pi_{ij} = \Pi_{ij} + \Delta\Pi_{ij}
\end{equation}
and assume for the moment that we have a perfect temporal integrator (this point will be addressed later).
We then solve
\begin{equation}
  \Box h_{ij} = \pi_{ij} = \Pi_{ij} + \Delta\Pi_{ij}
\end{equation}
so that the solution is
\begin{equation}
  h_{ij} = h_{ij}^{(0)} + \Delta h_{ij}
\end{equation}
where $h_{ij}^{(0)}$ is the ``true'' solution in the absence of discretisation errors.
We must then compute the only non-trivial rotational invariant of $\dot{h}_{ij}$
\begin{equation}
  \left|\dot{h}_{ij}\dot{h}^{ij}\right| = \mathrm{Tr}\left(\dot{h}^2\right)
\end{equation}
which we expand
\begin{equation}
  \left|\dot{h}_{ij}\dot{h}^{ij} \right|^2 = \left|\dot{h}_{ij}^{(0)}\dot{h}^{(0),ij}\right|^2 + 2\left|\dot{h}_{ij}^{(0)}\Delta\dot{h}^{ij}\right| + \left|\Delta\dot{h}_{ij}\Delta\dot{h}^{ij}\right|^2 \, .
\end{equation}
For stochastic or fully phase mixed sources, we might expect that the cross term to be small, but for the case of a coherent source, we do not expect the required phase cancellations to occur and the second term may be larger than the third term.  As well, while some numerical errors (such as machine roundoff) will enter in a random and uncorrelated way, numerical errors associated with numerical approximations (such as finite differencing) can instead exhibit high degrees of correlation, which can significantly bias final results (just as with any systematic error) as opposed to increasing the statistical uncertainty.

\subsection{Example: Numerical Leakage of Scalar}
Before turning to specific examples, let's consider how a field configuration $\phi$ leading to a purely scalar stress-energy tensor is artifically sourced by the combined effects of a discrete calculation of $T_{ij}$ and the subsequent TT projection.
As before, we concentrate on how these two effects modify an externally provided field configuration, rather than simultaneously considering the effects of the discretization in modifying the field evolution itself.
We will consider the general class of projectors
\begin{equation}
  \mathcal{O}_{ij,lm}
\end{equation}
For this entire class, it is straightforward to see that $\mathcal{O}_{ij,lm}\delta_{lm} = 0$.
We are therefore left to consider the projection of a stress-energy tensor of the form $T_{ij} = \partial_i\partial_jS$.
In the continuum, this gives
\begin{equation}
  T_{ij} = \partial_i\partial_j\nabla^{-4}\partial_l\partial_mT_{lm} = \partial_i\partial_j\nabla^{-4}(\nabla^2\phi)^2 \, .
\end{equation}
From this form we already see that numerous problems may arise if we a cavalier with our discretization choices.
The first issue that arises (see also below), is that $\partial_l\partial_mT_{lm}$ is no longer equal to $(\nabla^2\phi)^2$.

\subsection{Effects of Discrete Evolution of $\phi$}
\subsubsection{Linear Field Evolution}

\subsubsection{Comments on Nonlinear Field Evolution}
\jb{This is relevant to oscillons, since nonlinear field interactions hold them together.}
\jb{Hopefully understanding this makes it obvious why only the $g^2$ contribution below survives the modified projector.}

\section{Example: Gaussian Field Profile}
With the basic formalism in place, we will now compute the effects of numerical discretization on some simple field profiles motivated by the dynamics of nonlinear scalar fields.
Before turning to the question of dynamically evolving structure and exploring the issue of consistency between time evolution derivative stencils and projection operations,
here we first explore a simple non-dynamical toy model to qualitatively understand the projection of localized field structures into gravitational waves.

Here we consider the following field configuration
\begin{equation}
  \phi(x,y,z) = e^{-\sum\frac{x_i^2}{2\sigma_i^2}}
\end{equation}
with the widths $\sigma_i$ taken as free parameters.

First, we assume $\sigma_x=\sigma_y=\sigma_z=\sigma$.
In the continuum, it is straightforward to see that (here we drop the diagonal Lagrangian contribution)
\begin{subequations}
\begin{align}
  \Pi_{ij} = \frac{x_ix_j}{\sigma^4}e^{-x^2/\sigma^2}
\end{align}
\end{subequations}
with the remaining components easily obtained via the obvious substitutions.
Fourier transforming, we obtain
\begin{align}
  \tilde{\Pi}_{xx}({\bf k}) &= \pi^{3/2}\frac{\sigma}{2}e^{-\sigma^2k^2/4}\left(1-\frac{\sigma^2k_x^2}{2}\right) \\
  \tilde{\Pi}_{xy}({\bf k}) &= -\pi^{3/2}\frac{\sigma}{4}e^{-\sigma^2k^2/4}(\sigma k_x)(\sigma k_y)
\end{align}

For notational convenience, let's define
\begin{equation}
  \mathcal{A}(k) = \pi^{3/2}\sigma e^{-\frac{\sigma^2k^2}{4}}
\end{equation}
and combine the previous expressions
\begin{equation}
  \tilde{\Pi}_{ij} = \frac{\aSym}{2}\left(\delta_{ij} - \frac{\sigma^2k^2}{2}\hat{k}_i\hat{k}_j\right) \, .
\end{equation}
We have the following contractions of $\tilde{\Pi}_{lm}$
\begin{subequations}
\begin{align}
  \tilde{\Pi}_{ll} &= \frac{\aSym}{2}\left(3-\frac{\sigma^2k^2}{2}\right) \\
  \tilde{\Pi}_{ij}\tilde{\Pi}_{ji} &= \frac{\aSym^2}{2} + \left(\frac{\mathcal{A}}{2}\right)^2\left(1-\frac{\sigma^2k^2}{2}\right)^2 \\
  \hat{k}_l\tilde{\Pi}_{li} &= \frac{\aSym}{2}\hat{k}_i\left(1-\frac{\sigma^2k^2}{2}\right) \\
  \hat{k}_l\tilde{\Pi}_{lm}\hat{k}_m &= \frac{\aSym}{2}\left(1-\frac{\sigma^2k^2}{2}\right) \, .
\end{align}
\end{subequations}
We readily find that $\Pi_{ij}$ is purely scalar as expected, with
\begin{equation}
  \tilde{\Pi}_{ij}^{{\rm scalar}}({\bf k}) = \frac{\mathcal{A}(k)}{2}\left(\delta_{ij}-\hat{k}_i\hat{k}_j\frac{\sigma^2k^2}{2}\right) = \frac{\aSym}{2}\delta_{ij}\left(1-\frac{\sigma^2k^2}{6}\right) - \aSym\left(\hat{k}_i\hat{k}_j-\frac{1}{3}\delta_{ij}\right)\frac{\sigma^2 k^2}{4} \tilde{\Pi}_{ij} \, .
\end{equation}

Now, suppose that instead of computing exact derivatives, we approximate the first derivatives necessary to obtain the stress-tensor via
\begin{equation}
  D_x[\phi](x) = \frac{\sum_{\alpha}c_\alpha \phi}{2dx}
\end{equation}
with analogous definitions for the other directions.
In order to illustrate the issues, when performing analytic calculations we will take
\begin{equation}
  D_x[\phi](x) = \frac{1}{2dx}\left(\phi(x+dx,y,z)-\phi(x-dx,y,z)\right) \, .
\end{equation}
Let's denote the stress-tensor obtained this was as
\begin{equation}
  \pi_{ij} \equiv D_i[\phi]D_j[\phi] \, .
\end{equation}
Note that this choice of discretisation is already inconsistent with typical evolution schemes for both $\phi$ and $h_{ij}$ which often rely on second order accurate and second order isotropic Laplacian stencils.   However, the trace of $\pi_{ij}$ does not match the self-consistent definition of the isotropic pressure that is needed to evolve the Hubble constant.
Neverless, as our point is to illustrate how discretisation errors lead to numerical artifacts, we will proceed.
{\bf Insert a definition that's actually consistent with the trace being the isotropic pressure.}

Explicitly, for the second order accurate stencil~\eqref{eqn:} we have
\begin{align}
  \pi_{ij} &= \frac{e^{-\frac{x^2}{\sigma^2}}}{\sigma^2}e^{-\frac{dx^2}{\sigma^2}}\left(\frac{\sigma}{dx}\right)^2\sinh\left(\frac{x_idx}{\sigma^2}\right)\sinh\left(\frac{x_jdx}{\sigma^2}\right) 
\end{align}

Fourier transforming and using
\begin{equation}
  \int dx e^{ikx}e^{-x^2/\sigma^2}e^{Jx} = \sqrt{\pi\sigma^2}e^{\frac{\sigma^2 J^2}{4}}e^{-\frac{\sigma^2 k^2}{4}}e^{i\sigma^2kJ/2} \, ,
\end{equation}
we trivially obtain
\begin{subequations}
\begin{align}
  \tilde{\pi}_{xx}({\bf k}) &= \frac{\pi^{3/2}\sigma^3}{2}e^{-\sigma^2 k^2/4}\frac{1}{\sigma^2}\left(\frac{\sigma}{dx}\right)^2\left(1-e^{-\frac{dx^2}{\sigma^2}} - 2\sin^2\left(\frac{k_xdx}{2}\right)\right) \\
  &= \frac{\pi^{3/2}\sigma^3}{2}e^{-\sigma^2 k^2/4}\frac{1}{\sigma^2}\left(\frac{\sigma}{dx}\right)^2\left(\cos(k_xdx) - e^{-\frac{dx^2}{\sigma^2}}\right) \\
  \tilde{\pi}_{xy}({\bf k}) &= -\pi^{3/2}\sigma^3e^{-\frac{\sigma^2k^2}{4}}\frac{1}{\sigma^2}\left(\frac{\sigma}{dx}\right)^2\sin\left(\frac{k_xdx}{2}\right)\sin\left(\frac{k_ydx}{2}\right)e^{-\frac{dx^2}{2\sigma^2}}
\end{align}
\end{subequations}
We can combine these expressions as
\begin{equation}
  \tilde{\pi}_{ij} = \frac{\mathcal{A}}{2}\left(\frac{\sigma}{dx}\right)^2\delta_{ij}\left[\left(1-e^{-\frac{dx^2}{\sigma^2}}\right) - \frac{1}{2}\left(1-e^{-\frac{dx^2}{2\sigma^2}}\right)4\sin^2\left(\frac{k_idx}{2}\right)\right] - \frac{\mathcal{A}}{4}\left(\frac{\sigma}{dx}\right)^2e^{-\frac{dx^2}{2\sigma^2}}2\sin\left(\frac{k_idx}{2}\right)2\sin\left(\frac{k_jdx}{2}\right)
\end{equation}

Comparing the analytic result with the finite-difference approximation, we see that discretisation errors enter in two distinct ways: $k_idx$-dependent contributions resulting from the non-trivial Fourier space structure of the spatial stencils, and $\sigma^{-1}dx$ contributions arising directly from discretisation errors in the finite-difference derivatives.

To separate the effects of loss of resolution of the localised Gaussian versus distortion effects of the finite-differencing on large wavenumbers, it is convenient to introduce the following definitions
\begin{equation}
  f(x) \equiv \frac{1-e^{-2x}}{2x} = e^{-x}\frac{\sinh(x)}{x} \qquad g(x) \equiv 1-e^{-2x} = 2e^{-x}\sinh(x)
\end{equation}
which have the following limits
\begin{equation}
  f(0^+) = 1 \qquad g(0^+) = 0 \, .
\end{equation}
Further, for notational simplicity, we introduce the following effective wavevectors
\begin{subequations}
\begin{align}
  \kzero_i &\equiv \frac{2\sigma}{dx}\sin\left(\frac{k_idx}{2}\right) \approx \sigma k\hat{k}_i \\
  \ktwo_i &\equiv \hat{k}_i\frac{2\sigma}{dx}\sin\left(\frac{k_idx}{2}\right) = \hat{k}_i\kzero \approx \sigma k\hat{k}_i^2
\end{align}
\end{subequations}
where the approximate equalities hold for $k_idx \ll 1$.
We also will use the following natural definitions
\begin{equation}
  \kzero^2 \equiv \sum_i\kzero_i^2 = \left(\frac{2\sigma}{dx}\right)^2\sum_i\sin^2\left(\frac{k_idx}{2}\right) \approx \sigma^2k^2
\end{equation}
\begin{equation}
  \ktwo^2 \equiv \sum_i\ktwo_i^2 = \sum_i \hat{k}_i^2\left(\frac{2\sigma}{dx}\right)^2\sin^2\left(\frac{k_idx}{2}\right) \approx \sigma^2k^2\sum_i\hat{k}_i^4
\end{equation}
and
\begin{equation}
  \kone \equiv \sum_i\ktwo_i = \sum_i\hat{k}_i\frac{2\sigma}{dx}\sin\left(\frac{k_idx}{2}\right) \approx \sigma k\sum_i\hat{k}_i^2 = \sigma k
\end{equation}

We can now re-express $\tilde{\pi}_{ij}$
\begin{align}
  \tilde{\pi}_{ij} &= \frac{\aSym}{2}\delta_{ij}\left[f\left(\frac{dx^2}{2\sigma^2}\right) - \frac{1}{2}g\left(\frac{dx^2}{4\sigma^2}\right)\kzero_i\kzero_j\right] - \frac{\aSym}{4}e^{-\frac{dx^2}{2\sigma^2}}\kzero_i\kzero_j \\
  &= \frac{\aSym}{2}\left(\delta_{ij}f - \frac{1}{2}\kzero_i\kzero_j\right) + \frac{\aSym}{4}g\left(\frac{dx^2}{4\sigma^2}\right)\kzero_i\kzero_j\left(1-\delta_{ij}\right)
\end{align}
and
\begin{align}
  \tilde{\pi}_{il}\tilde{\pi}_{lj} &= \left(\frac{\aSym}{2}\right)^2\delta_{ij}\left(f^2 - fg\tilde{k}_i\tilde{k}_j + \frac{g^2}{4}\tilde{k}_i^2\tilde{k}_j^2\right) - \tilde{k}_i\tilde{k}_je^{-\frac{dx^2}{2\sigma^2}}\left(\frac{\aSym}{2}\right)^2\left( f- \frac{g}{4}\left(\tilde{k}_i^2+\tilde{k}_j^2\right) - \frac{e^{-\frac{dx^2}{2\sigma^2}}}{4}\tilde{k}^2\right) \\
  &= \left(\frac{\aSym}{2}\right)^2\left(\delta_{ij}f^2 - f\kzero_i\kzero_j+\frac{\kzero^2\kzero_i\kzero_j}{4}\right) + 
\end{align}

As before, we compute
\begin{subequations}
  \begin{align}
    \tilde{\pi}_{ii} &= \frac{\aSym}{2}\left(\frac{\sigma}{dx}\right)^2 \left[3\left(1-e^{-\frac{dx^2}{\sigma^2}}\right) - 2\sum_i\sin^2\left(\frac{k_idx}{2}\right)\right] = \frac{\mathcal{A}}{2}\left(3f\left(\frac{dx^2}{2\sigma^2}\right) - \frac{1}{2}\tilde{k}^2\right) \\
    \tilde{\pi}_{ij}\tilde{\pi}_{ji} &= \frac{\aSym^2}{2}f^2\left(\frac{dx^2}{2\sigma^2}\right) + \left(\frac{\aSym}{2}\right)^2\left(f\left(\frac{dx^2}{2\sigma^2}\right)-\frac{\tilde{k}^2}{2}\right)^2 - \left(\frac{\aSym}{4}\right)^2\left(1-e^{-\frac{dx^2}{\sigma^2}}\right)\left(\tilde{k}^4 - \sum_i\tilde{k}_i^4\right) \\
    \hat{k}_l\tilde{\pi}_{li} &= \hat{k}_i\frac{\aSym}{2}\left(f\left(\frac{dx^2}{2\sigma^2}\right)-\frac{1}{2}g\left(\frac{dx^2}{4\sigma^2}\right)\kzero_i^2\right) - \kzero_i\frac{\aSym}{4}e^{-\frac{dx^2}{2\sigma^2}}\kone = \left(\frac{\aSym}{2}\right)\left(\hat{k}_if - \frac{\kone\kzero_i}{2}\right) + \frac{\aSym}{4}g\left(\kone\kzero_i-\hat{k}_i\kzero_i^2\right) \\
    \hat{k}_l\tilde{\pi}_{lm}\hat{k}_m &= \frac{\aSym}{2}\left[f\left(\frac{dx^2}{2\sigma^2}\right) - \frac{1}{2}g\left(\frac{dx^2}{4\sigma^2}\right)\bar{k}^2 - \frac{e^{-\frac{dx^2}{2\sigma^2}}}{2}\left(\sum_l\bar{k}_l\right)^2 \right]  = \frac{\aSym}{2}\left[f\left(\frac{dx^2}{2\sigma^2}\right) - \frac{\kappa^2}{2} + \frac{1}{2}g\left(\frac{dx^2}{4\sigma^2}\right)\left(\kappa^2 - \bar{k}^2\right) \right] \\
    \hat{k}_i\tilde{\pi}_{il}\tilde{\pi}_{lj}\hat{k}_j &= \left(\frac{\aSym}{2}\right)^2\left(f^2 - f\kappa^2 + \frac{\kappa^2\tilde{k}^2}{4}\right) + \left(\frac{\aSym}{4}\right)^2g^2\left(\kone^2\kzero^2 - 2\kone\hat{k}_i\kzero_i^3 + \hat{k}_i^2\kzero_i^4 \right) + \left(\frac{\aSym}{2}\right)^2g\left(f\kone^2 - f\ktwo^2 - \frac{\kone^2\kzero^2}{2} + \frac{\kone\hat{k}_i\kzero_i^3}{2} \right) \notag \\
    &= \left(\frac{\aSym}{2}\right)^2\left(f^2 - f\kappa^2 + \frac{\kappa^2\tilde{k}^2}{4}\right) - \left(\frac{\aSym}{4}\right)^2\left(1-e^{-\frac{dx^2}{\sigma^2}}\right)\kone^2\kzero^2 + fg\left(\frac{\aSym}{2}\right)^2\left(\kone^2-\ktwo^2\right) + \left(\frac{\aSym}{4}\right)^2g^2\hat{k}_i^2\kzero_i^4 + \left(\frac{\aSym}{2}\right)^2e^{-\frac{dx^2}{2}}\frac{g}{2}\kone\hat{k}_i\kzero_i^3  \, .
  \end{align}
\end{subequations}
Combining, we find that $f$ drops out of the final calculation as required and we obtain
\begin{align}
  \left|\tilde{\pi}^{(T)}_{ij}\tilde{\pi}_{ji}^{(T)}\right| &= \left(\frac{\aSym}{2}\right)^2\frac{1}{8}\left(\kzero^2-\kone^2\right)
  - \left(\frac{\aSym}{4}\right)^2\left(1-e^{-\frac{dx^2}{\sigma^2}}\right)\left(\kzero^4+2\kone^2\kzero^2-\sum_i\kzero_i^4\right) \notag \\
  + g^2 \, .
\end{align}
Alternatively, we can directly obtain the TT projection
\begin{equation}
  \tilde{\pi}_{ij}^{(T)} = -\mathcal{\aSym}{4}\left(1-g\right)\left(\frac{dx^2}{4\sigma^2}\right)\left[\left(\kzero_i-\kone\hat{k}_i\right)\left(\kzero_j-\kone\hat{k}_j\right) - \frac{1}{2}\left(\delta_{ij}-\hat{k}_i\hat{k}_j\right)\left(\kzero^2-\kone^2\right)\right] + 
\end{equation}
from which we obtain
\begin{align}
  \left|\tilde{\pi}^{(T)}_{ij}\tilde{\pi}_{ji}^{(T)}\right| &= \frac{\aSym^2}{32}\left(1-g\right)^2\left(\kzero^2-\kone^2\right) \\
  &+ 2\left(\frac{\aSym}{4}\right)^2\left(\sum\kzero_i^4 + \kone^2\ktwo^2 - 2\kone\sum\kzero_i\ktwo_i - \frac{1}{2}\left(\kzero^2-\kone^2\right)\left(\kzero^2-\ktwo^2\right) \right)g(1-g) \\
  &+ \left(\frac{\aSym}{4}\right)^2\left(\right)g^2 \\
  &= 
  \left|\tilde{\pi}^{(T)}_{ij}\tilde{\pi}_{ji}^{(T)}\right| &= \frac{\aSym^2}{32}\left(1-g\right)^2\left(\kzero^2-\kone^2\right) \\
  &+ 2\left(\frac{\aSym}{4}\right)^2\left(\sum_l\left(\kzero_l^2-\kone\ktwo_l\right)^2 - \frac{1}{2}\left(\kzero^2-\kone^2\right)\left(\kzero^2-\ktwo^2\right) \right)g(1-g) \\
  &+ \left(\frac{\aSym}{4}\right)^2\left(\right)g^2
\end{align}
{\bf We note that in this specific case, using the modified projector $\tilde{P}_{ij}\equiv \delta_{ij} - \frac{\kzero_i\kzero_j}{\kzero^2}$ in our definition of the TT projector would elimnate all but the term proportional to $\mathcal{O}_{ij,lm}\kzero_i\kzero_j\kzero_l\kzero_m\delta_{ij}\delta{lm}$.  This is not entirely surprising as this definition is closely related to the corresponding Fourier transform of our finite-difference derivative.}

We can now compute the inferred spectrum of gravitational waves that would be produced by an oscillating spherically symmetric Gaussian profile using the standard combination of finite-differencing definition of the stress-energy tensor and continuum projectors.
This provides and important null test, since this configuration produces no gravitational waves.
The reader should not interpret this as a precise model for the oscillon dynamics (which even if spherically symmetric may not have a Gaussian radial profile), but rather an important test case to illustrate the shortcomings of the standard approach.
As well, we expect that up to $\mathcal{O}(1)$ corrections, this result should apply to the GW production from any spherically symmetric profile where the radial profile is well characterized by a single length scale.
We will examine how the introduction of a second scale (effectively a boundary layer between the interior and exterior) of the profile modifies this result shortly.

\begin{figure}
  \caption{Plotting the discrete version of the sum of a unit vector's components squared.  Show how absolutely awful it is once you get anywhere near the Nyquist.  In particular, find some notion of the asymmetry as a function of angle (maybe project onto various spherical harmonics?)}
\end{figure}


From this we obtain the various SVT components \emph{assuming the continuum based projection operators}.  We see that both vector and tensor contributions are generated
\begin{subequations}
  \begin{align}
    \tilde{\pi}_{ij}^{\rm scalar} &= \frac{1}{2}\delta_{ij}\mathcal{A}\left(f\left(\frac{dx^2}{2\sigma^2}\right) - g\left(\frac{dx^2}{4\sigma^2}\right)\bar{k}^2 - \frac{1}{4}\left[ \tilde{k}^2 - e^{-\frac{dx^2}{2\sigma^2}}(\sum \bar{k}_i)^2\right]\right) \\
       &- \frac{1}{2}\hat{k}_i\hat{k}_j\frac{\mathcal{A}}{4}\left(3e^{-\frac{dx^2}{2\sigma^2}}(\sum\bar{k}_i)^2 - \tilde{k}^2 + 3g\left(\frac{dx^2}{4\sigma^2}\right)\bar{k}^2 \right) \\
    \tilde{\pi}_{ij}^{(V)} &= \mathcal{A}\left(\frac{\sigma}{dx}\right)^2\frac{1}{2}\left[\hat{k}_i\hat{k}_j\sum_a\hat{k}_a2\sin\left(\frac{k_adx}{2}\right)\Sigma_a - \frac{1}{2}\left(\hat{k}_i2\sin\left(\frac{k_jdx}{2}\right)\Sigma_j + \hat{k}_j2\sin\left(\frac{k_idx}{2}\right)\Sigma_i\right) \right]\\
    \tilde{\pi}_{ij}^{(T)} &= \, .
  \end{align}
\end{subequations}

At this point, it is extremely unclear that a consistent set of projection operators associated with the given discrete definition of $\pi_{ij}$ can even be found.
This is not surprising, given that the trace of the operator as defined is inconsistent with a Laplacian stencil living on the lattice.
A better choice of discretisation should instead permit some notion of a discrete stencil to properly project out the unphysical numerical artifacts associated with the discrete stencils, which we will explore below.

\begin{figure}
  \includegraphics[width=3.375in]{{{txx-analytic}}}
  \includegraphics[width=3.375in]{{{txy-analytic}}} \\
  \includegraphics[width=3.375in]{{{txx-numerical-dx0.25}}}
  \includegraphics[width=3.375in]{{{txy-numerical-dx0.25}}} \\
  \includegraphics[width=3.375in]{{{txx-diff-dx0.25}}}
  \includegraphics[width=3.375in]{{{txy-diff-dx0.25}}}
  \caption{Comparision of the exact $\Pi_{ij}$ and the finite-difference $\pi_{ij}$ in real space for a single Gaussian.  We see that the numerical errors induce non-trivial structure into the errors with quadrupole and higher contributions.  We expect this to feed into errors in the calculation of gravitational wave energy densities.  In the top row we show the analytic result, in the middle row the numerical approximation, and in the bottom row the difference between the two.  In the left column we show $T_{xx}$ and in the right column $T_{xy}$.  The remaining components can be obtained trivially from these two.  In the case of the numerical stencil, we have taken $dx/\sigma = 0.25$.}
\end{figure}

\begin{figure}
  \includegraphics[width=3.375in]{{{convergence-max-error}}}
  \caption{Maximal real-space error in the calculation of $\Pi_{ij}$ using the finite-difference approximation as a function of the grid spacing $dx / \sigma$.}
\end{figure}

\begin{figure}
  \caption{Fourier space representations of the source.}
\end{figure}

\begin{figure}
  \caption{TT, scalar and vector projections of the sources based on a continuum projection operator.}
\end{figure}

\begin{figure}
  \caption{Total angular integral of TT component as a function of grid space $\sigma^{-1}dx$.}
\end{figure}


\section{Toy Model II: Asymmetric Gaussian}
We now consider a slightly more complex model, by allowing the Gaussian to be ellipsoidal with 3 different $\sigma_i$'s.
To simplify the algebra, we first consider the case where the principal axes of the ellipse lie along the coordinate axes.
The more general case allows rotation off of this configuration, which allows us to explore the influence of the anisotropy of the finite difference stencil.
{\bf Do this case as well.  Doesn't matter for analytic result, does for numerical.}
Carrying through the same calculations as above, we find

\subsection{Exact Result}
\begin{equation}
  \Pi_{ij} = \frac{x_ix_j}{\sigma_i^2\sigma_j^2}e^{-\sum\frac{x_i^2}{\sigma_i^2}}
\end{equation}
\begin{align}
  \tilde{\Pi}_{xx} &= \pi^{3/2}\left(\Pi_i\sigma_i\right)e^{-\frac{1}{4}\sum_i\sigma_i^2k_i^2}\frac{1}{2\sigma_x^2}\left(1-\frac{\sigma_x^2k_x^2}{2}\right) \\
  \tilde{\Pi}_{xy} &= -\pi^{3/2}\left(\Pi_i\sigma_i\right)e^{-\frac{1}{4}\sum_i\sigma_i^2k_i^2}\frac{1}{4\sigma_x\sigma_y}\sigma_xk_x\sigma_yk_y
\end{align}
\begin{equation}
  \tilde{\Pi}_{ij} = \pi^{3/2}(\sigma_x\sigma_y\sigma_z)e^{-\frac{1}{4}\sigma_i^2k_i^2}\frac{1}{2\sigma_i\sigma_j}\left(\delta_{ij} - \hat{k}_i\hat{k}_j\frac{\sigma_i\sigma_jk^2}{2}\right) \, .
\end{equation}

Define
\begin{equation}
  \aAsym = \pi^{3/2}\left(\Pi_i\sigma_i\right)e^{-\frac{1}{4}\sigma_i^2k_i^2}
\end{equation}

\begin{subequations}
\begin{align}
  \tilde{\Pi}_{ii} &= \frac{\aAsym}{2}\left(\sum_i\sigma_i^{-2} - \frac{k^2}{2}\right) \\
  \hat{k}_l\tilde{\Pi}_{li} &= \hat{k}_i\frac{\aAsym}{2\sigma_i^2}\left(1-\frac{\sigma_i^2k^2}{2}\right) \\
  \hat{k}_l\tilde{\Pi}_{lm}\hat{k}_m &= \frac{\aAsym}{2}\left[\sigma_a^{-2}\hat{k}_a^2 -\frac{k^2}{2}\right] 
\end{align}
\end{subequations}
\begin{align}
  \tilde{\Pi}_{ii} - \hat{k}_l\tilde{\Pi}_{lm}\hat{k}_m &= \frac{\aAsym}{2}\sum_i\frac{1-\hat{k}_i^2}{\sigma_i^2}\\
  3\hat{k}_l\tilde{\Pi}_{lm}\hat{k}_m - \tilde{\Pi}_{ii} &= -\aAsym\left(\sum_i\frac{3\hat{k}_i^2-1}{2\sigma_i^2} - \frac{k^2}{2}\right)
\end{align}
    {\bf Reexpress these in terms of basic matrix invariants, trace, trace of square and determinant}
    

From which we can easily compute the SVT decomposition
\begin{subequations}
  \begin{align}
    \Pi_{ij}^{\rm scalar} &= \frac{\delta_{ij}}{2}\aAsym\left(\sum_i\frac{1-\hat{k}_i^2}{2\sigma_i^2} \right) + \frac{\hat{k}_i\hat{k}_j}{2}\aAsym\left(\sum_i\frac{3\hat{k}_i^2-1}{2\sigma_i^2}-\frac{k^2}{2}\right)\\
    \Pi_{ij}^{(V)} &= \\
    \Pi_{ij}^{(T)} &=
  \end{align}
\end{subequations}


\subsection{Discrete Result}
\begin{equation}
  \pi_{ij} = \frac{e^{-\sum_i\frac{x_i^2}{\sigma_i^2}}}{\sigma_i\sigma_j}e^{-\frac{dx_i^2}{2\sigma_i^2}-\frac{dx_j^2}{2\sigma_j^2}}\frac{\sigma_i}{dx_i}\frac{\sigma_j}{dx_j}\sinh\left(\frac{x_idx_i}{\sigma_i^2}\right)\sinh\left(\frac{x_jdx_j}{\sigma_j^2}\right)
\end{equation}

\begin{equation}
  \tilde{\pi}_{ij}({\bf k}) = \pi^{3/2}(\Pi_i\sigma_i)e^{-\frac{1}{4}\sigma_i^2k_i^2}
\end{equation}
\begin{subequations}
\begin{align}
  \tilde{\pi}_{ii} &= \\
  \hat{k}_l\tilde{\pi}_{li} &= \\
  \hat{k}_l\tilde{\pi}_{lm}\hat{k}_m &= 
\end{align}
\end{subequations}


\section{Perturbations from Spherical Harmonic Perturbations to Sphericity}

\section{Toy Model: Smoothed Spherical Top-Hat}
Rather than the smoothly varying field concentration with only a single scale $\sigma$, we now consider a hierarchical concentration characterized by a boundary layer and an overall size
\begin{equation}
  \phi({\bf x}) = \tanh\left(w^{-1}[r-R_0]\right)
\end{equation}
with $w$ and $R_0$ parameters.
{\bf Perhaps better to use a breather profile to avoid singular derivative at the origin}.
Since this configuration is spherically symmetric, it should not produce any gravitational radiation.
However, we are particularly interested with how inadequate resolution of the boundary layer of width $w$ can source graviational waves, even if the overall radius of the structure $R_0$ is properly resolved.
{\bf With bubbles, for example, this effect modifies propagation speed of the expanding bubble wall, and leads quickly to a very visible effect simply from systematic numerical errors.}



\section{Discretisation Schemes}
\subsection{Off-Lattice Definitions}

\subsection{Action-Derivable Definitions}
Here we consider an alternative definition of the discrete operator $\pi_{ij}$, that can be derived from a discrete action formulation of the fields.
{\bf Put in action to quadratic order in $h_{ij}$, vary, show how this operator pops out}.
Suppose we use finite-difference Laplacian stencil
\begin{equation}
  L[\phi](x_i) \equiv \sum_\alpha c_\alpha\left(\phi_{i+\alpha}-\phi_i\right) \, .
\end{equation}
Once this is chosen, we are no longer free to choose the discretisation of $(\nabla\phi)^2$ or the discrete equations of motion will not follow from either the variation of a discrete action or from a discrete Hamiltonian.
This is obvious, since in the action only one differential operator $(\nabla\phi)^2$ appears, which means that $\nabla^2\phi$ is not independent of this definition.
The appropriate definition satisfying summation by parts is
\begin{equation}
  (\nabla\phi)^2(x_i) \to D_2[\phi](x_i) \equiv \sum_\alpha \frac{c_\alpha}{2}\left(\phi_{i+\alpha}-\phi_i\right)^2 \, .
\end{equation}
{\bf check signs.}
Based on this, it is reasonable to instead discretise as follows {\bf fix this so it's correct}
\begin{equation}
  \nabla_i\phi\nabla_j\phi(x_i) \to %\sum_{\alpha}\frac{c_\alpha}{2}\left(\phi_{i+\alpha}-\phi_i\right)\left(\phi_{i+}-\phi_i\right)
\end{equation}
so that we obtain
\begin{equation}
  \mathrm{Tr}(\partial_i\phi\partial_j\phi) = D_2[\phi](x_i)
\end{equation}
consistent with the definition required to conserve energy on the lattice.

\subsection{Symmetric Derivatives}
\begin{align}
  \tilde{\pi}_{xx}({\bf k}) &= \frac{\pi^{3/2}\sigma_x\sigma_y\sigma_z}{2}e^{-\frac{1}{4}\sum_i\sigma_I^2 k_i^2}\frac{1}{\sigma_x^2}\left(\frac{\sigma_x}{dx}\right)^2\left(\cos(k_xdx) - e^{-\frac{dx^2}{\sigma_z^2}}\right) \\
  \tilde{\pi}_{xy}({\bf k}) &= -\pi^{3/2}\left(\sigma_x\sigma_y\sigma_z\right)e^{-\frac{1}{4}\sum_i\sigma_I^2 k_i^2}\frac{1}{\sigma_x\sigma_y}\frac{\sigma_x}{dx}\sin\left(\frac{k_xdx}{2}\right)\frac{\sigma_y}{dy}\sin\left(\frac{k_ydy}{2}\right)e^{-\frac{dx^2}{4\sigma_x^2}-\frac{dy^2}{4\sigma_y^2}}
\end{align}

\section{Self-Consistent Action Derived Derivatives}


\section{Random Thoughts}
\begin{itemize}
\item If statistically we only get one or two oscillons per simulation volume, and you really think that the signal they produce is real, then you have to run many simulations in order to extract the signal produced by the oscillons themselves in order to fairly sample the random population of oscillons that will be produced.
\item This calculation is probably more efficient in spherical coordinates.  Do all the necessary tensor transforms.
\end{itemize}

\end{document}
